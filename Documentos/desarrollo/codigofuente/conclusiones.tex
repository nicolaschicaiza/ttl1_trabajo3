\section{Conclusiones}\label{sec:conclusiones}
\begin{itemize}
    \item Cuando a una señal cualquiera se le aplica un filtro cuya respuesta en frecuencia tiene una banda de paso. La cantidad de armónicos de la señal que no van ser atenuados dependerán del ancho de banda pasante que este tenga.
    \item Al aplicar un filtro con banda de paso variable (no constante) es importante conocer que a menor ancho de la banda de paso la densidad espectral de energía va a disminuir.
    \item La convolución es una operación capaz de modificar una señal a partir de otra y tiene la característica de ser unívoca, ya que al operar en un dominio va a tener una repercusión en el otro dominio.
    \item La presencia de la distorsión en señales que al filtrarlas con filtros cuyo ancho de banda es reducido va a presentar una mejor definición al permitir más componentes armónicas con y sin distorsión en comparación a un filtro ideal.
    \item El teorema de Rayleigh es una herramienta útil para realizar comparaciones a la respuesta de sistemas L.T.I., sin embargo, este teorema no permite analizar el comportamiento distorsión de un sistema.
    \item La transformada de Fourier facilita el trabajo de diseñar y operar con filtros, ya que es posible trabajar en el dominio en el cual las operaciones matemáticas sean mas simples de desarrollar.
    \item Una señal con ancho de banda infinito sufrirá de distorsión en algún punto de su espectro de magnitud si atraviesa un sistema con respuesta finita en frecuencia.
    \item Para realizar un análisis de distorsión de un filtro, es más fácil de visualizar en el dominio de la frecuencia que en el dominio del tiempo.
    \item La cantidad de escenarios de prueba que una simulación permite realizar es muy grande, sin embargo, los resultados mas relevantes se observan cuando se llevan los parámetros de simulación a valores limite (pequeños o grandes).
\end{itemize}