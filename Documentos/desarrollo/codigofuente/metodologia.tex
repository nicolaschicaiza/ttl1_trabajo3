\section{Metodología}\label{metodologia}
El proceso que se llevó a cabo para obtener los resultados se realizó en el lenguaje MATLAB a través de su IDE Oficial, además se identificó una relación en los procesos de los dos primeros ítems llevando a considerar crear un único script para todas las actividades del proyecto, así tener la posibilidad de reutilizar código.

En primer lugar, se identificó los segmentos del trabajo 2 donde se hacen los respectivos cálculos y se realiza la Transformada de Fourier para la señal planteada con el fin de adaptarlos para el objetivo de este proyecto. Una vez obtenida la función de la señal planteada se generó el diagrama de bloques para tener claridad de los cálculos que se deben aplicar.

\begin{figure}[b!]
	\centering
	\begin{tikzpicture}[circuit ee IEC]
		% Mixer
		\node[draw,
			circle,
			minimum size=0.6cm,
		] (mix) at (0,0){};

		\draw (mix.north east) -- (mix.south west)
		(mix.north west) -- (mix.south east);

		\draw (mix.north east) -- (mix.south west)
		(mix.north west) -- (mix.south east);

		% LPF
		\node[draw,
			%fill=Goldenrod,
			%minimum width=2cm,
			%minimum height=1.2cm,
			right=1cm of mix
		]   (lpf) {LPF};

		% Local Oscillator
		\node [draw,
			ac source,
			minimum size=0.6cm,
			label=below:$cos(2\pi f_c t)$
			%below right= 1cm and -1.19cm of mix
		]  (lo) at (0,-1.5) {};

		% Arrows with text label 
		\draw[-stealth] (mix.east) -- (lpf.west)
		node[midway,above]{};

		\draw[-stealth] (lpf.east) -- ++ (1.25,0)
		node[midway](output){}node[right]{$y(t)$};

		\draw[stealth-] (mix.south) -- (lo.north)
		node[below,right] {};

		\draw[stealth-] (mix.west) -- ++(-1,0)
		node[left]{$x(t)$};
	\end{tikzpicture}
	\caption{Caption}
	\label{fig:my_label}
\end{figure}

\vspace{-5mm}

El diagrama que se consideró para el ítem número 1 es el mostrado en la figura \ref{fig:diagramabloques1}, que corresponde a la secuencia de tener una operación de desplazamiento en el caso de tener una señal en la región de pasa banda y la operación de aplicar un filtro en banda base \cite{filtros2021}. Pasando a la parte de desarrollo del funcionamiento se construyó la función del filtro pasa bajas directamente en el dominio de la frecuencia para luego proceder hacer la multiplicación con la Transformada de Fourier de la función de la señal asignada.

Para la segunda parte del proyecto se realizó una investigación y revisión a los documentos guía de la asignatura para planificar la construcción del filtro trapezoidal, en ellos se menciona el uso de la convolución para modificar una señal a partir de otra señal y realizan una demostración con la convolución de dos pulsos rectangulares de dimensiones diferentes que obtiene como resultado una función en forma de trapezoide \cite{ejercicio2021}. Sin embargo, en la teoría de la convolución la modificación que sufra una de las señales involucradas automáticamente va ha generar una modificación en la respuesta de la convolución, por ello es importante identificar sus parámetros para determinar que cambios va a sufrir la respuesta al modificar uno de sus parámetros \cite{Chaparro2013}.

% Graphic for TeX using PGF
% Title: /home/jnicolaschc/GitHub/Teoría de Telecominicaciones I /ttl1_trabajo3/Documentos/desarrollo/codigofuente/pgf/ventanasInicial.dia
% Creator: Dia v0.97+git
% CreationDate: Fri Aug 20 23:11:49 2021
% For: jnicolaschc
% \usepackage{tikz}
% The following commands are not supported in PSTricks at present
% We define them conditionally, so when they are implemented,
% this pgf file will use them.
\begin{figure}[]
	\centering
	\ifx\du\undefined
		\newlength{\du}
	\fi
	\setlength{\du}{15\unitlength}
	\begin{tikzpicture}[scale = 1]
		\pgftransformxscale{1.000000}
		\pgftransformyscale{-1.000000}
		\definecolor{dialinecolor}{rgb}{0.000000, 0.000000, 0.000000}
		\pgfsetstrokecolor{dialinecolor}
		\pgfsetstrokeopacity{1.000000}
		\definecolor{diafillcolor}{rgb}{1.000000, 1.000000, 1.000000}
		\pgfsetfillcolor{diafillcolor}
		\pgfsetfillopacity{1.000000}
		\pgfsetlinewidth{0.050000\du}
		\pgfsetdash{}{0pt}
		\pgfsetbuttcap
		{
			\definecolor{diafillcolor}{rgb}{0.000000, 0.000000, 0.000000}
			\pgfsetfillcolor{diafillcolor}
			\pgfsetfillopacity{1.000000}
			% was here!!!
			\pgfsetarrowsend{stealth}
			\definecolor{dialinecolor}{rgb}{0.000000, 0.000000, 0.000000}
			\pgfsetstrokecolor{dialinecolor}
			\pgfsetstrokeopacity{1.000000}
			\draw (33.016000\du,15.011800\du)--(33.027500\du,9.028390\du);
		}
		\pgfsetlinewidth{0.050000\du}
		\pgfsetdash{}{0pt}
		\pgfsetbuttcap
		{
			\definecolor{diafillcolor}{rgb}{0.000000, 0.000000, 0.000000}
			\pgfsetfillcolor{diafillcolor}
			\pgfsetfillopacity{1.000000}
			% was here!!!
			\pgfsetarrowsend{stealth}
			\definecolor{dialinecolor}{rgb}{0.000000, 0.000000, 0.000000}
			\pgfsetstrokecolor{dialinecolor}
			\pgfsetstrokeopacity{1.000000}
			\draw (30.001400\du,15.000400\du)--(36.993500\du,15.011800\du);
		}
		\pgfsetlinewidth{0.040000\du}
		\pgfsetdash{}{0pt}
		\pgfsetbuttcap
		{
			\definecolor{diafillcolor}{rgb}{0.000000, 0.000000, 0.000000}
			\pgfsetfillcolor{diafillcolor}
			\pgfsetfillopacity{1.000000}
			% was here!!!
			\definecolor{dialinecolor}{rgb}{0.000000, 0.000000, 0.000000}
			\pgfsetstrokecolor{dialinecolor}
			\pgfsetstrokeopacity{1.000000}
			\draw (31.010100\du,15.011800\du)--(31.010100\du,13.028800\du);
		}
		\pgfsetlinewidth{0.040000\du}
		\pgfsetdash{}{0pt}
		\pgfsetbuttcap
		{
			\definecolor{diafillcolor}{rgb}{0.000000, 0.000000, 0.000000}
			\pgfsetfillcolor{diafillcolor}
			\pgfsetfillopacity{1.000000}
			% was here!!!
			\definecolor{dialinecolor}{rgb}{0.000000, 0.000000, 0.000000}
			\pgfsetstrokecolor{dialinecolor}
			\pgfsetstrokeopacity{1.000000}
			\draw (34.993800\du,14.986000\du)--(34.993800\du,13.002900\du);
		}
		\pgfsetlinewidth{0.040000\du}
		\pgfsetdash{}{0pt}
		\pgfsetbuttcap
		{
			\definecolor{diafillcolor}{rgb}{0.000000, 0.000000, 0.000000}
			\pgfsetfillcolor{diafillcolor}
			\pgfsetfillopacity{1.000000}
			% was here!!!
			\definecolor{dialinecolor}{rgb}{0.000000, 0.000000, 0.000000}
			\pgfsetstrokecolor{dialinecolor}
			\pgfsetstrokeopacity{1.000000}
			\draw (31.010100\du,13.034500\du)--(34.987600\du,13.011600\du);
		}
		% setfont left to latex
		\definecolor{dialinecolor}{rgb}{0.000000, 0.000000, 0.000000}
		\pgfsetstrokecolor{dialinecolor}
		\pgfsetstrokeopacity{1.000000}
		\definecolor{diafillcolor}{rgb}{0.000000, 0.000000, 0.000000}
		\pgfsetfillcolor{diafillcolor}
		\pgfsetfillopacity{1.000000}
		\node[anchor=base,inner sep=0pt, outer sep=0pt,color=dialinecolor] at (30.998600\du,15.471197\du){\scriptsize -a};
		% setfont left to latex
		\definecolor{dialinecolor}{rgb}{0.000000, 0.000000, 0.000000}
		\pgfsetstrokecolor{dialinecolor}
		\pgfsetstrokeopacity{1.000000}
		\definecolor{diafillcolor}{rgb}{0.000000, 0.000000, 0.000000}
		\pgfsetfillcolor{diafillcolor}
		\pgfsetfillopacity{1.000000}
		\node[anchor=base,inner sep=0pt, outer sep=0pt,color=dialinecolor] at (34.989700\du,15.458054\du){\scriptsize a};
		% setfont left to latex
		\definecolor{dialinecolor}{rgb}{0.000000, 0.000000, 0.000000}
		\pgfsetstrokecolor{dialinecolor}
		\pgfsetstrokeopacity{1.000000}
		\definecolor{diafillcolor}{rgb}{0.000000, 0.000000, 0.000000}
		\pgfsetfillcolor{diafillcolor}
		\pgfsetfillopacity{1.000000}
		\node[anchor=base,inner sep=0pt, outer sep=0pt,color=dialinecolor] at (33.018100\du,15.474264\du){\scriptsize 0};
		% setfont left to latex
		\definecolor{dialinecolor}{rgb}{0.000000, 0.000000, 0.000000}
		\pgfsetstrokecolor{dialinecolor}
		\pgfsetstrokeopacity{1.000000}
		\definecolor{diafillcolor}{rgb}{0.000000, 0.000000, 0.000000}
		\pgfsetfillcolor{diafillcolor}
		\pgfsetfillopacity{1.000000}
		\node[anchor=base,inner sep=0pt, outer sep=0pt,color=dialinecolor] at (33.018100\du,15.897598\du){};
		% setfont left to latex
		\definecolor{dialinecolor}{rgb}{0.000000, 0.000000, 0.000000}
		\pgfsetstrokecolor{dialinecolor}
		\pgfsetstrokeopacity{1.000000}
		\definecolor{diafillcolor}{rgb}{0.000000, 0.000000, 0.000000}
		\pgfsetfillcolor{diafillcolor}
		\pgfsetfillopacity{1.000000}
		\node[anchor=base east,inner sep=0pt, outer sep=0pt,color=dialinecolor] at (30.894967\du,13.132469\du){\scriptsize 1};
		% setfont left to latex
		\definecolor{dialinecolor}{rgb}{0.000000, 0.000000, 0.000000}
		\pgfsetstrokecolor{dialinecolor}
		\pgfsetstrokeopacity{1.000000}
		\definecolor{diafillcolor}{rgb}{0.000000, 0.000000, 0.000000}
		\pgfsetfillcolor{diafillcolor}
		\pgfsetfillopacity{1.000000}
		\node[anchor=base,inner sep=0pt, outer sep=0pt,color=dialinecolor] at (36.555500\du,14.643575\du){\scriptsize t};
		% setfont left to latex
		\definecolor{dialinecolor}{rgb}{0.000000, 0.000000, 0.000000}
		\pgfsetstrokecolor{dialinecolor}
		\pgfsetstrokeopacity{1.000000}
		\definecolor{diafillcolor}{rgb}{0.000000, 0.000000, 0.000000}
		\pgfsetfillcolor{diafillcolor}
		\pgfsetfillopacity{1.000000}
		\node[anchor=base west,inner sep=0pt,outer sep=0pt,color=dialinecolor] at (33.350682\du,9.813504\du){\scriptsize x(t)};
		\pgfsetlinewidth{0.050000\du}
		\pgfsetdash{}{0pt}
		\pgfsetbuttcap
		{
			\definecolor{diafillcolor}{rgb}{1.000000, 0.000000, 0.000000}
			\pgfsetfillcolor{diafillcolor}
			\pgfsetfillopacity{1.000000}
			% was here!!!
			\pgfsetarrowsend{stealth}
			\definecolor{dialinecolor}{rgb}{1.000000, 0.000000, 0.000000}
			\pgfsetstrokecolor{dialinecolor}
			\pgfsetstrokeopacity{1.000000}
			\draw (41.979500\du,15.042600\du)--(41.991000\du,9.059170\du);
		}
		\pgfsetlinewidth{0.050000\du}
		\pgfsetdash{}{0pt}
		\pgfsetbuttcap
		{
			\definecolor{diafillcolor}{rgb}{1.000000, 0.000000, 0.000000}
			\pgfsetfillcolor{diafillcolor}
			\pgfsetfillopacity{1.000000}
			% was here!!!
			\pgfsetarrowsend{stealth}
			\definecolor{dialinecolor}{rgb}{1.000000, 0.000000, 0.000000}
			\pgfsetstrokecolor{dialinecolor}
			\pgfsetstrokeopacity{1.000000}
			\draw (38.047900\du,15.031100\du)--(46.003100\du,15.006100\du);
		}
		\pgfsetlinewidth{0.040000\du}
		\pgfsetdash{}{0pt}
		\pgfsetbuttcap
		{
			\definecolor{diafillcolor}{rgb}{1.000000, 0.000000, 0.000000}
			\pgfsetfillcolor{diafillcolor}
			\pgfsetfillopacity{1.000000}
			% was here!!!
			\definecolor{dialinecolor}{rgb}{1.000000, 0.000000, 0.000000}
			\pgfsetstrokecolor{dialinecolor}
			\pgfsetstrokeopacity{1.000000}
			\draw (39.480700\du,15.042600\du)--(39.480700\du,13.059600\du);
		}
		\pgfsetlinewidth{0.040000\du}
		\pgfsetdash{}{0pt}
		\pgfsetbuttcap
		{
			\definecolor{diafillcolor}{rgb}{1.000000, 0.000000, 0.000000}
			\pgfsetfillcolor{diafillcolor}
			\pgfsetfillopacity{1.000000}
			% was here!!!
			\definecolor{dialinecolor}{rgb}{1.000000, 0.000000, 0.000000}
			\pgfsetstrokecolor{dialinecolor}
			\pgfsetstrokeopacity{1.000000}
			\draw (44.496100\du,15.016700\du)--(44.496100\du,13.033700\du);
		}
		\pgfsetlinewidth{0.040000\du}
		\pgfsetdash{}{0pt}
		\pgfsetbuttcap
		{
			\definecolor{diafillcolor}{rgb}{1.000000, 0.000000, 0.000000}
			\pgfsetfillcolor{diafillcolor}
			\pgfsetfillopacity{1.000000}
			% was here!!!
			\definecolor{dialinecolor}{rgb}{1.000000, 0.000000, 0.000000}
			\pgfsetstrokecolor{dialinecolor}
			\pgfsetstrokeopacity{1.000000}
			\draw (39.480700\du,13.065300\du)--(44.490000\du,13.034500\du);
		}
		% setfont left to latex
		\definecolor{dialinecolor}{rgb}{1.000000, 0.000000, 0.000000}
		\pgfsetstrokecolor{dialinecolor}
		\pgfsetstrokeopacity{1.000000}
		\definecolor{diafillcolor}{rgb}{1.000000, 0.000000, 0.000000}
		\pgfsetfillcolor{diafillcolor}
		\pgfsetfillopacity{1.000000}
		\node[anchor=base,inner sep=0pt, outer sep=0pt,color=dialinecolor] at (39.469300\du,15.492359\du){\scriptsize -b};
		% setfont left to latex
		\definecolor{dialinecolor}{rgb}{1.000000, 0.000000, 0.000000}
		\pgfsetstrokecolor{dialinecolor}
		\pgfsetstrokeopacity{1.000000}
		\definecolor{diafillcolor}{rgb}{1.000000, 0.000000, 0.000000}
		\pgfsetfillcolor{diafillcolor}
		\pgfsetfillopacity{1.000000}
		\node[anchor=base,inner sep=0pt, outer sep=0pt,color=dialinecolor] at (44.503400\du,15.492359\du){\scriptsize b};
		% setfont left to latex
		\definecolor{dialinecolor}{rgb}{1.000000, 0.000000, 0.000000}
		\pgfsetstrokecolor{dialinecolor}
		\pgfsetstrokeopacity{1.000000}
		\definecolor{diafillcolor}{rgb}{1.000000, 0.000000, 0.000000}
		\pgfsetfillcolor{diafillcolor}
		\pgfsetfillopacity{1.000000}
		\node[anchor=base,inner sep=0pt, outer sep=0pt,color=dialinecolor] at (41.970200\du,15.492359\du){\scriptsize 0};
		% setfont left to latex
		\definecolor{dialinecolor}{rgb}{1.000000, 0.000000, 0.000000}
		\pgfsetstrokecolor{dialinecolor}
		\pgfsetstrokeopacity{1.000000}
		\definecolor{diafillcolor}{rgb}{1.000000, 0.000000, 0.000000}
		\pgfsetfillcolor{diafillcolor}
		\pgfsetfillopacity{1.000000}
		\node[anchor=base,inner sep=0pt, outer sep=0pt,color=dialinecolor] at (41.970200\du,15.915692\du){};
		% setfont left to latex
		\definecolor{dialinecolor}{rgb}{1.000000, 0.000000, 0.000000}
		\pgfsetstrokecolor{dialinecolor}
		\pgfsetstrokeopacity{1.000000}
		\definecolor{diafillcolor}{rgb}{1.000000, 0.000000, 0.000000}
		\pgfsetfillcolor{diafillcolor}
		\pgfsetfillopacity{1.000000}
		\node[anchor=base east,inner sep=0pt, outer sep=0pt,color=dialinecolor] at (39.346290\du,13.134352\du){\scriptsize 1};
		% setfont left to latex
		\definecolor{dialinecolor}{rgb}{1.000000, 0.000000, 0.000000}
		\pgfsetstrokecolor{dialinecolor}
		\pgfsetstrokeopacity{1.000000}
		\definecolor{diafillcolor}{rgb}{1.000000, 0.000000, 0.000000}
		\pgfsetfillcolor{diafillcolor}
		\pgfsetfillopacity{1.000000}
		\node[anchor=base,inner sep=0pt, outer sep=0pt,color=dialinecolor] at (45.579922\du,14.613913\du){\scriptsize t};
		% setfont left to latex
		\definecolor{dialinecolor}{rgb}{1.000000, 0.000000, 0.000000}
		\pgfsetstrokecolor{dialinecolor}
		\pgfsetstrokeopacity{1.000000}
		\definecolor{diafillcolor}{rgb}{1.000000, 0.000000, 0.000000}
		\pgfsetfillcolor{diafillcolor}
		\pgfsetfillopacity{1.000000}
		\node[anchor=base west,inner sep=0pt,outer sep=0pt,color=dialinecolor] at (42.368392\du,9.805729\du){\scriptsize h(t)};
		% setfont left to latex
		\definecolor{dialinecolor}{rgb}{0.000000, 0.000000, 0.000000}
		\pgfsetstrokecolor{dialinecolor}
		\pgfsetstrokeopacity{1.000000}
		\definecolor{diafillcolor}{rgb}{0.000000, 0.000000, 0.000000}
		\pgfsetfillcolor{diafillcolor}
		\pgfsetfillopacity{1.000000}
		\node[anchor=base west,inner sep=0pt,outer sep=0pt,color=dialinecolor] at (37.589600\du,10.730600\du){\scriptsize Donde b>a};
	\end{tikzpicture}
	\caption{}
	\label{}
\end{figure}

\vspace{-5mm}

Como se observa en la figura \ref{fig:ventanasIniciales} se tienen dos pulsos rectangulares cuyo ancho son diferentes con el fin de cumplir una condición necesaria para generar la función en forma de trapezoide, para ello el proceso de la convolución se puede ver como un traslape entre las dos funciones involucradas, este proceso se puede observar en la siguiente secuencia de figuras.

\vspace{-3mm}
% Graphic for TeX using PGF
% Title: /home/jnicolaschc/GitHub/Teoría de Telecominicaciones I /ttl1_trabajo3/Documentos/desarrollo/codigofuente/pgf/traslape1.dia
% Creator: Dia v0.97+git
% CreationDate: Fri Aug 20 23:46:44 2021
% For: jnicolaschc
% \usepackage{tikz}
% The following commands are not supported in PSTricks at present
% We define them conditionally, so when they are implemented,
% this pgf file will use them.
\begin{figure}[H]
	\centering
	\ifx\du\undefined
		\newlength{\du}
	\fi
	\setlength{\du}{15\unitlength}
	\begin{tikzpicture}[scale = 0.8]
		\pgftransformxscale{1.000000}
		\pgftransformyscale{-1.000000}
		\definecolor{dialinecolor}{rgb}{0.000000, 0.000000, 0.000000}
		\pgfsetstrokecolor{dialinecolor}
		\pgfsetstrokeopacity{1.000000}
		\definecolor{diafillcolor}{rgb}{1.000000, 1.000000, 1.000000}
		\pgfsetfillcolor{diafillcolor}
		\pgfsetfillopacity{1.000000}
		\pgfsetlinewidth{0.050000\du}
		\pgfsetdash{}{0pt}
		\pgfsetbuttcap
		{
			\definecolor{diafillcolor}{rgb}{0.000000, 0.000000, 0.000000}
			\pgfsetfillcolor{diafillcolor}
			\pgfsetfillopacity{1.000000}
			% was here!!!
			\pgfsetarrowsend{stealth}
			\definecolor{dialinecolor}{rgb}{0.000000, 0.000000, 0.000000}
			\pgfsetstrokecolor{dialinecolor}
			\pgfsetstrokeopacity{1.000000}
			\draw (40.002568\du,15.011800\du)--(40.014068\du,9.028390\du);
		}
		\pgfsetlinewidth{0.050000\du}
		\pgfsetdash{}{0pt}
		\pgfsetbuttcap
		{
			\definecolor{diafillcolor}{rgb}{0.000000, 0.000000, 0.000000}
			\pgfsetfillcolor{diafillcolor}
			\pgfsetfillopacity{1.000000}
			% was here!!!
			\pgfsetarrowsend{stealth}
			\definecolor{dialinecolor}{rgb}{0.000000, 0.000000, 0.000000}
			\pgfsetstrokecolor{dialinecolor}
			\pgfsetstrokeopacity{1.000000}
			\draw (34.004392\du,15.022957\du)--(43.996117\du,14.995695\du);
		}
		\pgfsetlinewidth{0.040000\du}
		\pgfsetdash{}{0pt}
		\pgfsetbuttcap
		{
			\definecolor{diafillcolor}{rgb}{0.000000, 0.000000, 0.000000}
			\pgfsetfillcolor{diafillcolor}
			\pgfsetfillopacity{1.000000}
			% was here!!!
			\definecolor{dialinecolor}{rgb}{0.000000, 0.000000, 0.000000}
			\pgfsetstrokecolor{dialinecolor}
			\pgfsetstrokeopacity{1.000000}
			\draw (37.996668\du,15.011800\du)--(37.996668\du,13.028800\du);
		}
		\pgfsetlinewidth{0.040000\du}
		\pgfsetdash{}{0pt}
		\pgfsetbuttcap
		{
			\definecolor{diafillcolor}{rgb}{0.000000, 0.000000, 0.000000}
			\pgfsetfillcolor{diafillcolor}
			\pgfsetfillopacity{1.000000}
			% was here!!!
			\definecolor{dialinecolor}{rgb}{0.000000, 0.000000, 0.000000}
			\pgfsetstrokecolor{dialinecolor}
			\pgfsetstrokeopacity{1.000000}
			\draw (41.980368\du,14.986000\du)--(41.980368\du,13.002900\du);
		}
		\pgfsetlinewidth{0.040000\du}
		\pgfsetdash{}{0pt}
		\pgfsetbuttcap
		{
			\definecolor{diafillcolor}{rgb}{0.000000, 0.000000, 0.000000}
			\pgfsetfillcolor{diafillcolor}
			\pgfsetfillopacity{1.000000}
			% was here!!!
			\definecolor{dialinecolor}{rgb}{0.000000, 0.000000, 0.000000}
			\pgfsetstrokecolor{dialinecolor}
			\pgfsetstrokeopacity{1.000000}
			\draw (37.996668\du,13.034500\du)--(42.011054\du,13.038374\du);
		}
		% setfont left to latex
		\definecolor{dialinecolor}{rgb}{0.000000, 0.000000, 0.000000}
		\pgfsetstrokecolor{dialinecolor}
		\pgfsetstrokeopacity{1.000000}
		\definecolor{diafillcolor}{rgb}{0.000000, 0.000000, 0.000000}
		\pgfsetfillcolor{diafillcolor}
		\pgfsetfillopacity{1.000000}
		\node[anchor=base,inner sep=0pt, outer sep=0pt,color=dialinecolor] at (38.000033\du,15.471175\du){\tiny -a};
		% setfont left to latex
		\definecolor{dialinecolor}{rgb}{0.000000, 0.000000, 0.000000}
		\pgfsetstrokecolor{dialinecolor}
		\pgfsetstrokeopacity{1.000000}
		\definecolor{diafillcolor}{rgb}{0.000000, 0.000000, 0.000000}
		\pgfsetfillcolor{diafillcolor}
		\pgfsetfillopacity{1.000000}
		\node[anchor=base,inner sep=0pt, outer sep=0pt,color=dialinecolor] at (41.991133\du,15.458075\du){\tiny a};
		% setfont left to latex
		\definecolor{dialinecolor}{rgb}{0.000000, 0.000000, 0.000000}
		\pgfsetstrokecolor{dialinecolor}
		\pgfsetstrokeopacity{1.000000}
		\definecolor{diafillcolor}{rgb}{0.000000, 0.000000, 0.000000}
		\pgfsetfillcolor{diafillcolor}
		\pgfsetfillopacity{1.000000}
		\node[anchor=base east,inner sep=0pt, outer sep=0pt,color=dialinecolor] at (39.799158\du,12.850010\du){\tiny 1};
		% setfont left to latex
		\definecolor{dialinecolor}{rgb}{0.000000, 0.000000, 0.000000}
		\pgfsetstrokecolor{dialinecolor}
		\pgfsetstrokeopacity{1.000000}
		\definecolor{diafillcolor}{rgb}{0.000000, 0.000000, 0.000000}
		\pgfsetfillcolor{diafillcolor}
		\pgfsetfillopacity{1.000000}
		\node[anchor=base,inner sep=0pt, outer sep=0pt,color=dialinecolor] at (43.532653\du,14.548198\du){\tiny $\tau$};
		% setfont left to latex
		\definecolor{dialinecolor}{rgb}{0.000000, 0.000000, 0.000000}
		\pgfsetstrokecolor{dialinecolor}
		\pgfsetstrokeopacity{1.000000}
		\definecolor{diafillcolor}{rgb}{0.000000, 0.000000, 0.000000}
		\pgfsetfillcolor{diafillcolor}
		\pgfsetfillopacity{1.000000}
		\node[anchor=base west,inner sep=0pt,outer sep=0pt,color=dialinecolor] at (40.337268\du,9.823361\du){\tiny x($\tau$)};
		\pgfsetlinewidth{0.040000\du}
		\pgfsetdash{}{0pt}
		\pgfsetbuttcap
		{
			\definecolor{diafillcolor}{rgb}{1.000000, 0.000000, 0.000000}
			\pgfsetfillcolor{diafillcolor}
			\pgfsetfillopacity{1.000000}
			% was here!!!
			\definecolor{dialinecolor}{rgb}{1.000000, 0.000000, 0.000000}
			\pgfsetstrokecolor{dialinecolor}
			\pgfsetstrokeopacity{1.000000}
			\draw (35.467140\du,15.042600\du)--(35.467140\du,13.059600\du);
		}
		\pgfsetlinewidth{0.040000\du}
		\pgfsetdash{}{0pt}
		\pgfsetbuttcap
		{
			\definecolor{diafillcolor}{rgb}{1.000000, 0.000000, 0.000000}
			\pgfsetfillcolor{diafillcolor}
			\pgfsetfillopacity{1.000000}
			% was here!!!
			\definecolor{dialinecolor}{rgb}{1.000000, 0.000000, 0.000000}
			\pgfsetstrokecolor{dialinecolor}
			\pgfsetstrokeopacity{1.000000}
			\draw (40.482540\du,15.016700\du)--(40.482540\du,13.033700\du);
		}
		\pgfsetlinewidth{0.040000\du}
		\pgfsetdash{}{0pt}
		\pgfsetbuttcap
		{
			\definecolor{diafillcolor}{rgb}{1.000000, 0.000000, 0.000000}
			\pgfsetfillcolor{diafillcolor}
			\pgfsetfillopacity{1.000000}
			% was here!!!
			\definecolor{dialinecolor}{rgb}{1.000000, 0.000000, 0.000000}
			\pgfsetstrokecolor{dialinecolor}
			\pgfsetstrokeopacity{1.000000}
			\draw (35.467140\du,13.065300\du)--(40.476440\du,13.034500\du);
		}
		% setfont left to latex
		\definecolor{dialinecolor}{rgb}{1.000000, 0.000000, 0.000000}
		\pgfsetstrokecolor{dialinecolor}
		\pgfsetstrokeopacity{1.000000}
		\definecolor{diafillcolor}{rgb}{1.000000, 0.000000, 0.000000}
		\pgfsetfillcolor{diafillcolor}
		\pgfsetfillopacity{1.000000}
		\node[anchor=base,inner sep=0pt, outer sep=0pt,color=dialinecolor] at (35.455740\du,15.502231\du){\tiny t-b};
		% setfont left to latex
		\definecolor{dialinecolor}{rgb}{1.000000, 0.000000, 0.000000}
		\pgfsetstrokecolor{dialinecolor}
		\pgfsetstrokeopacity{1.000000}
		\definecolor{diafillcolor}{rgb}{1.000000, 0.000000, 0.000000}
		\pgfsetfillcolor{diafillcolor}
		\pgfsetfillopacity{1.000000}
		\node[anchor=base,inner sep=0pt, outer sep=0pt,color=dialinecolor] at (40.489840\du,15.502231\du){\tiny t-b};
		% setfont left to latex
		\definecolor{dialinecolor}{rgb}{1.000000, 0.000000, 0.000000}
		\pgfsetstrokecolor{dialinecolor}
		\pgfsetstrokeopacity{1.000000}
		\definecolor{diafillcolor}{rgb}{1.000000, 0.000000, 0.000000}
		\pgfsetfillcolor{diafillcolor}
		\pgfsetfillopacity{1.000000}
		\node[anchor=base east,inner sep=0pt, outer sep=0pt,color=dialinecolor] at (35.332740\du,13.134346\du){\tiny 1};
		% setfont left to latex
		\definecolor{dialinecolor}{rgb}{1.000000, 0.000000, 0.000000}
		\pgfsetstrokecolor{dialinecolor}
		\pgfsetstrokeopacity{1.000000}
		\definecolor{diafillcolor}{rgb}{1.000000, 0.000000, 0.000000}
		\pgfsetfillcolor{diafillcolor}
		\pgfsetfillopacity{1.000000}
		\node[anchor=base west,inner sep=0pt,outer sep=0pt,color=dialinecolor] at (35.723728\du,11.406141\du){\tiny h(t-$\tau$)};
		% setfont left to latex
		\definecolor{dialinecolor}{rgb}{1.000000, 0.000000, 0.000000}
		\pgfsetstrokecolor{dialinecolor}
		\pgfsetstrokeopacity{1.000000}
		\definecolor{diafillcolor}{rgb}{1.000000, 0.000000, 0.000000}
		\pgfsetfillcolor{diafillcolor}
		\pgfsetfillopacity{1.000000}
		\node[anchor=base,inner sep=0pt, outer sep=0pt,color=dialinecolor] at (43.715072\du,15.482150\du){\tiny t};
		\pgfsetlinewidth{0.030000\du}
		\pgfsetdash{{0.200000\du}{0.200000\du}}{0\du}
		\pgfsetmiterjoin
		\pgfsetbuttcap
		{\pgfsetcornersarced{\pgfpoint{0.000000\du}{0.000000\du}}\definecolor{diafillcolor}{rgb}{0.498039, 0.498039, 0.498039}
			\pgfsetfillcolor{diafillcolor}
			\pgfsetfillopacity{1.000000}
			\fill (38.017810\du,13.053239\du)--(38.017810\du,14.977525\du)--(40.465090\du,14.977525\du)--(40.465090\du,13.053239\du)--cycle;
		}{\pgfsetcornersarced{\pgfpoint{0.000000\du}{0.000000\du}}\definecolor{dialinecolor}{rgb}{0.000000, 0.000000, 0.000000}
			\pgfsetstrokecolor{dialinecolor}
			\pgfsetstrokeopacity{1.000000}
			\draw (38.017810\du,13.053239\du)--(38.017810\du,14.977525\du)--(40.465090\du,14.977525\du)--(40.465090\du,13.053239\du)--cycle;
		}\pgfsetlinewidth{0.050000\du}
		\pgfsetdash{}{0pt}
		\pgfsetbuttcap
		{
			\definecolor{diafillcolor}{rgb}{0.000000, 0.000000, 0.000000}
			\pgfsetfillcolor{diafillcolor}
			\pgfsetfillopacity{1.000000}
			% was here!!!
			\pgfsetarrowsend{stealth}
			\definecolor{dialinecolor}{rgb}{0.000000, 0.000000, 0.000000}
			\pgfsetstrokecolor{dialinecolor}
			\pgfsetstrokeopacity{1.000000}
			\draw (44.986279\du,15.023266\du)--(54.978004\du,14.996003\du);
		}
		\pgfsetlinewidth{0.050000\du}
		\pgfsetdash{}{0pt}
		\pgfsetbuttcap
		{
			\definecolor{diafillcolor}{rgb}{0.000000, 0.000000, 0.000000}
			\pgfsetfillcolor{diafillcolor}
			\pgfsetfillopacity{1.000000}
			% was here!!!
			\pgfsetarrowsend{stealth}
			\definecolor{dialinecolor}{rgb}{0.000000, 0.000000, 0.000000}
			\pgfsetstrokecolor{dialinecolor}
			\pgfsetstrokeopacity{1.000000}
			\draw (49.982142\du,15.009634\du)--(49.993878\du,9.052459\du);
		}
		\pgfsetlinewidth{0.050000\du}
		\pgfsetdash{{0.300000\du}{0.120000\du}{0.060000\du}{0.120000\du}}{0cm}
		\pgfsetbuttcap
		{
			\definecolor{diafillcolor}{rgb}{0.101961, 0.682353, 0.623529}
			\pgfsetfillcolor{diafillcolor}
			\pgfsetfillopacity{1.000000}
			% was here!!!
			\definecolor{dialinecolor}{rgb}{0.101961, 0.682353, 0.623529}
			\pgfsetstrokecolor{dialinecolor}
			\pgfsetstrokeopacity{1.000000}
			\draw (46.996410\du,15.021052\du)--(47.008146\du,9.063877\du);
		}
		\pgfsetlinewidth{0.050000\du}
		\pgfsetdash{{0.300000\du}{0.120000\du}{0.060000\du}{0.120000\du}}{0cm}
		\pgfsetbuttcap
		{
			\definecolor{diafillcolor}{rgb}{0.101961, 0.682353, 0.623529}
			\pgfsetfillcolor{diafillcolor}
			\pgfsetfillopacity{1.000000}
			% was here!!!
			\definecolor{dialinecolor}{rgb}{0.101961, 0.682353, 0.623529}
			\pgfsetstrokecolor{dialinecolor}
			\pgfsetstrokeopacity{1.000000}
			\draw (48.980224\du,15.007421\du)--(48.991960\du,9.050246\du);
		}
		\pgfsetlinewidth{0.050000\du}
		\pgfsetdash{{0.300000\du}{0.120000\du}{0.060000\du}{0.120000\du}}{0cm}
		\pgfsetbuttcap
		{
			\definecolor{diafillcolor}{rgb}{0.101961, 0.682353, 0.623529}
			\pgfsetfillcolor{diafillcolor}
			\pgfsetfillopacity{1.000000}
			% was here!!!
			\definecolor{dialinecolor}{rgb}{0.101961, 0.682353, 0.623529}
			\pgfsetstrokecolor{dialinecolor}
			\pgfsetstrokeopacity{1.000000}
			\draw (51.022189\du,14.980158\du)--(51.033925\du,9.022983\du);
		}
		\pgfsetlinewidth{0.050000\du}
		\pgfsetdash{{0.300000\du}{0.120000\du}{0.060000\du}{0.120000\du}}{0cm}
		\pgfsetbuttcap
		{
			\definecolor{diafillcolor}{rgb}{0.101961, 0.682353, 0.623529}
			\pgfsetfillcolor{diafillcolor}
			\pgfsetfillopacity{1.000000}
			% was here!!!
			\definecolor{dialinecolor}{rgb}{0.101961, 0.682353, 0.623529}
			\pgfsetstrokecolor{dialinecolor}
			\pgfsetstrokeopacity{1.000000}
			\draw (52.995998\du,14.980158\du)--(53.007734\du,9.022983\du);
		}
		\pgfsetlinewidth{0.060000\du}
		\pgfsetdash{}{0pt}
		\pgfsetbuttcap
		{
			\definecolor{diafillcolor}{rgb}{0.396078, 0.345098, 0.960784}
			\pgfsetfillcolor{diafillcolor}
			\pgfsetfillopacity{1.000000}
			% was here!!!
			\definecolor{dialinecolor}{rgb}{0.396078, 0.345098, 0.960784}
			\pgfsetstrokecolor{dialinecolor}
			\pgfsetstrokeopacity{1.000000}
			\draw (45.005231\du,15.011197\du)--(46.988241\du,15.011197\du);
		}
		\pgfsetlinewidth{0.060000\du}
		\pgfsetdash{}{0pt}
		\pgfsetbuttcap
		{
			\definecolor{diafillcolor}{rgb}{0.396078, 0.345098, 0.960784}
			\pgfsetfillcolor{diafillcolor}
			\pgfsetfillopacity{1.000000}
			% was here!!!
			\definecolor{dialinecolor}{rgb}{0.396078, 0.345098, 0.960784}
			\pgfsetstrokecolor{dialinecolor}
			\pgfsetstrokeopacity{1.000000}
			\draw (47.011166\du,14.999735\du)--(48.352276\du,12.122651\du);
		}
		% setfont left to latex
		\definecolor{dialinecolor}{rgb}{0.396078, 0.345098, 0.960784}
		\pgfsetstrokecolor{dialinecolor}
		\pgfsetstrokeopacity{1.000000}
		\definecolor{diafillcolor}{rgb}{0.396078, 0.345098, 0.960784}
		\pgfsetfillcolor{diafillcolor}
		\pgfsetfillopacity{1.000000}
		\node[anchor=base,inner sep=0pt, outer sep=0pt,color=dialinecolor] at (47.009480\du,15.461516\du){\tiny -(a+b)};
		% setfont left to latex
		\definecolor{dialinecolor}{rgb}{0.396078, 0.345098, 0.960784}
		\pgfsetstrokecolor{dialinecolor}
		\pgfsetstrokeopacity{1.000000}
		\definecolor{diafillcolor}{rgb}{0.396078, 0.345098, 0.960784}
		\pgfsetfillcolor{diafillcolor}
		\pgfsetfillopacity{1.000000}
		\node[anchor=base,inner sep=0pt, outer sep=0pt,color=dialinecolor] at (48.976569\du,15.450054\du){\tiny a-b};
		% setfont left to latex
		\definecolor{dialinecolor}{rgb}{0.396078, 0.345098, 0.960784}
		\pgfsetstrokecolor{dialinecolor}
		\pgfsetstrokeopacity{1.000000}
		\definecolor{diafillcolor}{rgb}{0.396078, 0.345098, 0.960784}
		\pgfsetfillcolor{diafillcolor}
		\pgfsetfillopacity{1.000000}
		\node[anchor=base,inner sep=0pt, outer sep=0pt,color=dialinecolor] at (51.037524\du,15.451660\du){\tiny b-a};
		% setfont left to latex
		\definecolor{dialinecolor}{rgb}{0.396078, 0.345098, 0.960784}
		\pgfsetstrokecolor{dialinecolor}
		\pgfsetstrokeopacity{1.000000}
		\definecolor{diafillcolor}{rgb}{0.396078, 0.345098, 0.960784}
		\pgfsetfillcolor{diafillcolor}
		\pgfsetfillopacity{1.000000}
		\node[anchor=base,inner sep=0pt, outer sep=0pt,color=dialinecolor] at (53.006778\du,15.451660\du){\tiny b+a};
		% setfont left to latex
		\definecolor{dialinecolor}{rgb}{0.000000, 0.000000, 0.000000}
		\pgfsetstrokecolor{dialinecolor}
		\pgfsetstrokeopacity{1.000000}
		\definecolor{diafillcolor}{rgb}{0.000000, 0.000000, 0.000000}
		\pgfsetfillcolor{diafillcolor}
		\pgfsetfillopacity{1.000000}
		\node[anchor=base,inner sep=0pt, outer sep=0pt,color=dialinecolor] at (54.525875\du,14.580511\du){\tiny t};
		% setfont left to latex
		\definecolor{dialinecolor}{rgb}{0.000000, 0.000000, 0.000000}
		\pgfsetstrokecolor{dialinecolor}
		\pgfsetstrokeopacity{1.000000}
		\definecolor{diafillcolor}{rgb}{0.000000, 0.000000, 0.000000}
		\pgfsetfillcolor{diafillcolor}
		\pgfsetfillopacity{1.000000}
		\node[anchor=base,inner sep=0pt, outer sep=0pt,color=dialinecolor] at (50.420842\du,10.820816\du){\tiny 2a};
		% setfont left to latex
		\definecolor{dialinecolor}{rgb}{0.396078, 0.345098, 0.960784}
		\pgfsetstrokecolor{dialinecolor}
		\pgfsetstrokeopacity{1.000000}
		\definecolor{diafillcolor}{rgb}{0.396078, 0.345098, 0.960784}
		\pgfsetfillcolor{diafillcolor}
		\pgfsetfillopacity{1.000000}
		\node[anchor=base,inner sep=0pt, outer sep=0pt,color=dialinecolor] at (50.558392\du,9.743342\du){\tiny y(t)};
	\end{tikzpicture}
	\vspace{-2mm}
	\caption{\scriptsize Traslape de banda de transición izquierda.}
	\label{fig:traslape1}
\end{figure}

\vspace{-9mm}
% Graphic for TeX using PGF
% Title: /home/jnicolaschc/GitHub/Teoría de Telecominicaciones I /ttl1_trabajo3/Documentos/desarrollo/codigofuente/pgf/traslape2.dia
% Creator: Dia v0.97+git
% CreationDate: Sat Aug 21 00:25:50 2021
% For: jnicolaschc
% \usepackage{tikz}
% The following commands are not supported in PSTricks at present
% We define them conditionally, so when they are implemented,
% this pgf file will use them.
\begin{figure}[H]
	\centering
	\ifx\du\undefined
		\newlength{\du}
	\fi
	\setlength{\du}{15\unitlength}
	\begin{tikzpicture}[scale = 0.8]
		\pgftransformxscale{1.000000}
		\pgftransformyscale{-1.000000}
		\definecolor{dialinecolor}{rgb}{0.000000, 0.000000, 0.000000}
		\pgfsetstrokecolor{dialinecolor}
		\pgfsetstrokeopacity{1.000000}
		\definecolor{diafillcolor}{rgb}{1.000000, 1.000000, 1.000000}
		\pgfsetfillcolor{diafillcolor}
		\pgfsetfillopacity{1.000000}
		\pgfsetlinewidth{0.050000\du}
		\pgfsetdash{}{0pt}
		\pgfsetbuttcap
		{
			\definecolor{diafillcolor}{rgb}{0.000000, 0.000000, 0.000000}
			\pgfsetfillcolor{diafillcolor}
			\pgfsetfillopacity{1.000000}
			% was here!!!
			\pgfsetarrowsend{stealth}
			\definecolor{dialinecolor}{rgb}{0.000000, 0.000000, 0.000000}
			\pgfsetstrokecolor{dialinecolor}
			\pgfsetstrokeopacity{1.000000}
			\draw (40.002600\du,15.011800\du)--(40.014100\du,9.028390\du);
		}
		\pgfsetlinewidth{0.050000\du}
		\pgfsetdash{}{0pt}
		\pgfsetbuttcap
		{
			\definecolor{diafillcolor}{rgb}{0.000000, 0.000000, 0.000000}
			\pgfsetfillcolor{diafillcolor}
			\pgfsetfillopacity{1.000000}
			% was here!!!
			\pgfsetarrowsend{stealth}
			\definecolor{dialinecolor}{rgb}{0.000000, 0.000000, 0.000000}
			\pgfsetstrokecolor{dialinecolor}
			\pgfsetstrokeopacity{1.000000}
			\draw (34.004400\du,15.023000\du)--(43.996100\du,14.995700\du);
		}
		\pgfsetlinewidth{0.040000\du}
		\pgfsetdash{}{0pt}
		\pgfsetbuttcap
		{
			\definecolor{diafillcolor}{rgb}{0.000000, 0.000000, 0.000000}
			\pgfsetfillcolor{diafillcolor}
			\pgfsetfillopacity{1.000000}
			% was here!!!
			\definecolor{dialinecolor}{rgb}{0.000000, 0.000000, 0.000000}
			\pgfsetstrokecolor{dialinecolor}
			\pgfsetstrokeopacity{1.000000}
			\draw (37.996700\du,15.011800\du)--(37.996700\du,13.028800\du);
		}
		\pgfsetlinewidth{0.040000\du}
		\pgfsetdash{}{0pt}
		\pgfsetbuttcap
		{
			\definecolor{diafillcolor}{rgb}{0.000000, 0.000000, 0.000000}
			\pgfsetfillcolor{diafillcolor}
			\pgfsetfillopacity{1.000000}
			% was here!!!
			\definecolor{dialinecolor}{rgb}{0.000000, 0.000000, 0.000000}
			\pgfsetstrokecolor{dialinecolor}
			\pgfsetstrokeopacity{1.000000}
			\draw (41.980400\du,14.986000\du)--(41.980400\du,13.002900\du);
		}
		\pgfsetlinewidth{0.040000\du}
		\pgfsetdash{}{0pt}
		\pgfsetbuttcap
		{
			\definecolor{diafillcolor}{rgb}{0.000000, 0.000000, 0.000000}
			\pgfsetfillcolor{diafillcolor}
			\pgfsetfillopacity{1.000000}
			% was here!!!
			\definecolor{dialinecolor}{rgb}{0.000000, 0.000000, 0.000000}
			\pgfsetstrokecolor{dialinecolor}
			\pgfsetstrokeopacity{1.000000}
			\draw (37.996700\du,13.034500\du)--(42.011100\du,13.038400\du);
		}
		% setfont left to latex
		\definecolor{dialinecolor}{rgb}{0.000000, 0.000000, 0.000000}
		\pgfsetstrokecolor{dialinecolor}
		\pgfsetstrokeopacity{1.000000}
		\definecolor{diafillcolor}{rgb}{0.000000, 0.000000, 0.000000}
		\pgfsetfillcolor{diafillcolor}
		\pgfsetfillopacity{1.000000}
		\node[anchor=base,inner sep=0pt, outer sep=0pt,color=dialinecolor] at (38.000000\du,15.471175\du){\tiny -a};
		% setfont left to latex
		\definecolor{dialinecolor}{rgb}{0.000000, 0.000000, 0.000000}
		\pgfsetstrokecolor{dialinecolor}
		\pgfsetstrokeopacity{1.000000}
		\definecolor{diafillcolor}{rgb}{0.000000, 0.000000, 0.000000}
		\pgfsetfillcolor{diafillcolor}
		\pgfsetfillopacity{1.000000}
		\node[anchor=base east,inner sep=0pt, outer sep=0pt,color=dialinecolor] at (39.799200\du,12.850046\du){\tiny 1};
		% setfont left to latex
		\definecolor{dialinecolor}{rgb}{0.000000, 0.000000, 0.000000}
		\pgfsetstrokecolor{dialinecolor}
		\pgfsetstrokeopacity{1.000000}
		\definecolor{diafillcolor}{rgb}{0.000000, 0.000000, 0.000000}
		\pgfsetfillcolor{diafillcolor}
		\pgfsetfillopacity{1.000000}
		\node[anchor=base,inner sep=0pt, outer sep=0pt,color=dialinecolor] at (43.532700\du,14.528504\du){\tiny $\tau	$};
		% setfont left to latex
		\definecolor{dialinecolor}{rgb}{0.000000, 0.000000, 0.000000}
		\pgfsetstrokecolor{dialinecolor}
		\pgfsetstrokeopacity{1.000000}
		\definecolor{diafillcolor}{rgb}{0.000000, 0.000000, 0.000000}
		\pgfsetfillcolor{diafillcolor}
		\pgfsetfillopacity{1.000000}
		\node[anchor=base west,inner sep=0pt,outer sep=0pt,color=dialinecolor] at (40.337300\du,9.823361\du){\tiny x($\tau$)};
		\pgfsetlinewidth{0.040000\du}
		\pgfsetdash{}{0pt}
		\pgfsetbuttcap
		{
			\definecolor{diafillcolor}{rgb}{1.000000, 0.000000, 0.000000}
			\pgfsetfillcolor{diafillcolor}
			\pgfsetfillopacity{1.000000}
			% was here!!!
			\definecolor{dialinecolor}{rgb}{1.000000, 0.000000, 0.000000}
			\pgfsetstrokecolor{dialinecolor}
			\pgfsetstrokeopacity{1.000000}
			\draw (37.429316\du,15.042600\du)--(37.429316\du,13.059600\du);
		}
		\pgfsetlinewidth{0.040000\du}
		\pgfsetdash{}{0pt}
		\pgfsetbuttcap
		{
			\definecolor{diafillcolor}{rgb}{1.000000, 0.000000, 0.000000}
			\pgfsetfillcolor{diafillcolor}
			\pgfsetfillopacity{1.000000}
			% was here!!!
			\definecolor{dialinecolor}{rgb}{1.000000, 0.000000, 0.000000}
			\pgfsetstrokecolor{dialinecolor}
			\pgfsetstrokeopacity{1.000000}
			\draw (42.444716\du,15.016700\du)--(42.444716\du,13.033700\du);
		}
		\pgfsetlinewidth{0.040000\du}
		\pgfsetdash{}{0pt}
		\pgfsetbuttcap
		{
			\definecolor{diafillcolor}{rgb}{1.000000, 0.000000, 0.000000}
			\pgfsetfillcolor{diafillcolor}
			\pgfsetfillopacity{1.000000}
			% was here!!!
			\definecolor{dialinecolor}{rgb}{1.000000, 0.000000, 0.000000}
			\pgfsetstrokecolor{dialinecolor}
			\pgfsetstrokeopacity{1.000000}
			\draw (37.411638\du,13.065300\du)--(42.420938\du,13.034500\du);
		}
		% setfont left to latex
		\definecolor{dialinecolor}{rgb}{1.000000, 0.000000, 0.000000}
		\pgfsetstrokecolor{dialinecolor}
		\pgfsetstrokeopacity{1.000000}
		\definecolor{diafillcolor}{rgb}{1.000000, 0.000000, 0.000000}
		\pgfsetfillcolor{diafillcolor}
		\pgfsetfillopacity{1.000000}
		\node[anchor=base,inner sep=0pt, outer sep=0pt,color=dialinecolor] at (37.453271\du,15.492375\du){\tiny t-b};
		% setfont left to latex
		\definecolor{dialinecolor}{rgb}{1.000000, 0.000000, 0.000000}
		\pgfsetstrokecolor{dialinecolor}
		\pgfsetstrokeopacity{1.000000}
		\definecolor{diafillcolor}{rgb}{1.000000, 0.000000, 0.000000}
		\pgfsetfillcolor{diafillcolor}
		\pgfsetfillopacity{1.000000}
		\node[anchor=base,inner sep=0pt, outer sep=0pt,color=dialinecolor] at (42.487371\du,15.492375\du){\tiny t-b};
		% setfont left to latex
		\definecolor{dialinecolor}{rgb}{1.000000, 0.000000, 0.000000}
		\pgfsetstrokecolor{dialinecolor}
		\pgfsetstrokeopacity{1.000000}
		\definecolor{diafillcolor}{rgb}{1.000000, 0.000000, 0.000000}
		\pgfsetfillcolor{diafillcolor}
		\pgfsetfillopacity{1.000000}
		\node[anchor=base east,inner sep=0pt, outer sep=0pt,color=dialinecolor] at (37.330271\du,13.134346\du){\tiny 1};
		% setfont left to latex
		\definecolor{dialinecolor}{rgb}{1.000000, 0.000000, 0.000000}
		\pgfsetstrokecolor{dialinecolor}
		\pgfsetstrokeopacity{1.000000}
		\definecolor{diafillcolor}{rgb}{1.000000, 0.000000, 0.000000}
		\pgfsetfillcolor{diafillcolor}
		\pgfsetfillopacity{1.000000}
		\node[anchor=base west,inner sep=0pt,outer sep=0pt,color=dialinecolor] at (38.110178\du,11.406131\du){\tiny h(t-$\tau$)};
		% setfont left to latex
		\definecolor{dialinecolor}{rgb}{1.000000, 0.000000, 0.000000}
		\pgfsetstrokecolor{dialinecolor}
		\pgfsetstrokeopacity{1.000000}
		\definecolor{diafillcolor}{rgb}{1.000000, 0.000000, 0.000000}
		\pgfsetfillcolor{diafillcolor}
		\pgfsetfillopacity{1.000000}
		\node[anchor=base,inner sep=0pt, outer sep=0pt,color=dialinecolor] at (43.715100\du,15.482117\du){\tiny t};
		\pgfsetlinewidth{0.030000\du}
		\pgfsetdash{{0.200000\du}{0.200000\du}}{0\du}
		\pgfsetmiterjoin
		\pgfsetbuttcap
		{\pgfsetcornersarced{\pgfpoint{0.000000\du}{0.000000\du}}\definecolor{diafillcolor}{rgb}{0.498039, 0.498039, 0.498039}
			\pgfsetfillcolor{diafillcolor}
			\pgfsetfillopacity{1.000000}
			\fill (38.015698\du,13.053200\du)--(38.015698\du,14.977486\du)--(41.932323\du,14.977486\du)--(41.932323\du,13.053200\du)--cycle;
		}{\pgfsetcornersarced{\pgfpoint{0.000000\du}{0.000000\du}}\definecolor{dialinecolor}{rgb}{0.000000, 0.000000, 0.000000}
			\pgfsetstrokecolor{dialinecolor}
			\pgfsetstrokeopacity{1.000000}
			\draw (38.015698\du,13.053200\du)--(38.015698\du,14.977486\du)--(41.932323\du,14.977486\du)--(41.932323\du,13.053200\du)--cycle;
		}\pgfsetlinewidth{0.050000\du}
		\pgfsetdash{}{0pt}
		\pgfsetbuttcap
		{
			\definecolor{diafillcolor}{rgb}{0.000000, 0.000000, 0.000000}
			\pgfsetfillcolor{diafillcolor}
			\pgfsetfillopacity{1.000000}
			% was here!!!
			\pgfsetarrowsend{stealth}
			\definecolor{dialinecolor}{rgb}{0.000000, 0.000000, 0.000000}
			\pgfsetstrokecolor{dialinecolor}
			\pgfsetstrokeopacity{1.000000}
			\draw (44.986300\du,15.023300\du)--(54.978000\du,14.996000\du);
		}
		\pgfsetlinewidth{0.050000\du}
		\pgfsetdash{}{0pt}
		\pgfsetbuttcap
		{
			\definecolor{diafillcolor}{rgb}{0.000000, 0.000000, 0.000000}
			\pgfsetfillcolor{diafillcolor}
			\pgfsetfillopacity{1.000000}
			% was here!!!
			\pgfsetarrowsend{stealth}
			\definecolor{dialinecolor}{rgb}{0.000000, 0.000000, 0.000000}
			\pgfsetstrokecolor{dialinecolor}
			\pgfsetstrokeopacity{1.000000}
			\draw (49.982100\du,15.009600\du)--(49.993900\du,9.052460\du);
		}
		\pgfsetlinewidth{0.050000\du}
		\pgfsetdash{{0.300000\du}{0.120000\du}{0.060000\du}{0.120000\du}}{0cm}
		\pgfsetbuttcap
		{
			\definecolor{diafillcolor}{rgb}{0.101961, 0.682353, 0.623529}
			\pgfsetfillcolor{diafillcolor}
			\pgfsetfillopacity{1.000000}
			% was here!!!
			\definecolor{dialinecolor}{rgb}{0.101961, 0.682353, 0.623529}
			\pgfsetstrokecolor{dialinecolor}
			\pgfsetstrokeopacity{1.000000}
			\draw (46.996400\du,15.021100\du)--(47.008100\du,9.063880\du);
		}
		\pgfsetlinewidth{0.050000\du}
		\pgfsetdash{{0.300000\du}{0.120000\du}{0.060000\du}{0.120000\du}}{0cm}
		\pgfsetbuttcap
		{
			\definecolor{diafillcolor}{rgb}{0.101961, 0.682353, 0.623529}
			\pgfsetfillcolor{diafillcolor}
			\pgfsetfillopacity{1.000000}
			% was here!!!
			\definecolor{dialinecolor}{rgb}{0.101961, 0.682353, 0.623529}
			\pgfsetstrokecolor{dialinecolor}
			\pgfsetstrokeopacity{1.000000}
			\draw (48.980200\du,15.007400\du)--(48.992000\du,9.050250\du);
		}
		\pgfsetlinewidth{0.050000\du}
		\pgfsetdash{{0.300000\du}{0.120000\du}{0.060000\du}{0.120000\du}}{0cm}
		\pgfsetbuttcap
		{
			\definecolor{diafillcolor}{rgb}{0.101961, 0.682353, 0.623529}
			\pgfsetfillcolor{diafillcolor}
			\pgfsetfillopacity{1.000000}
			% was here!!!
			\definecolor{dialinecolor}{rgb}{0.101961, 0.682353, 0.623529}
			\pgfsetstrokecolor{dialinecolor}
			\pgfsetstrokeopacity{1.000000}
			\draw (51.022200\du,14.980200\du)--(51.033900\du,9.022980\du);
		}
		\pgfsetlinewidth{0.050000\du}
		\pgfsetdash{{0.300000\du}{0.120000\du}{0.060000\du}{0.120000\du}}{0cm}
		\pgfsetbuttcap
		{
			\definecolor{diafillcolor}{rgb}{0.101961, 0.682353, 0.623529}
			\pgfsetfillcolor{diafillcolor}
			\pgfsetfillopacity{1.000000}
			% was here!!!
			\definecolor{dialinecolor}{rgb}{0.101961, 0.682353, 0.623529}
			\pgfsetstrokecolor{dialinecolor}
			\pgfsetstrokeopacity{1.000000}
			\draw (52.996000\du,14.980200\du)--(53.007700\du,9.022980\du);
		}
		\pgfsetlinewidth{0.060000\du}
		\pgfsetdash{}{0pt}
		\pgfsetbuttcap
		{
			\definecolor{diafillcolor}{rgb}{0.396078, 0.345098, 0.960784}
			\pgfsetfillcolor{diafillcolor}
			\pgfsetfillopacity{1.000000}
			% was here!!!
			\definecolor{dialinecolor}{rgb}{0.396078, 0.345098, 0.960784}
			\pgfsetstrokecolor{dialinecolor}
			\pgfsetstrokeopacity{1.000000}
			\draw (45.005200\du,15.011200\du)--(46.988200\du,15.011200\du);
		}
		\pgfsetlinewidth{0.060000\du}
		\pgfsetdash{}{0pt}
		\pgfsetbuttcap
		{
			\definecolor{diafillcolor}{rgb}{0.396078, 0.345098, 0.960784}
			\pgfsetfillcolor{diafillcolor}
			\pgfsetfillopacity{1.000000}
			% was here!!!
			\definecolor{dialinecolor}{rgb}{0.396078, 0.345098, 0.960784}
			\pgfsetstrokecolor{dialinecolor}
			\pgfsetstrokeopacity{1.000000}
			\draw (47.011200\du,14.999700\du)--(48.958144\du,10.988487\du);
		}
		% setfont left to latex
		\definecolor{dialinecolor}{rgb}{0.396078, 0.345098, 0.960784}
		\pgfsetstrokecolor{dialinecolor}
		\pgfsetstrokeopacity{1.000000}
		\definecolor{diafillcolor}{rgb}{0.396078, 0.345098, 0.960784}
		\pgfsetfillcolor{diafillcolor}
		\pgfsetfillopacity{1.000000}
		\node[anchor=base,inner sep=0pt, outer sep=0pt,color=dialinecolor] at (47.009500\du,15.451675\du){\tiny -(a+b)};
		% setfont left to latex
		\definecolor{dialinecolor}{rgb}{0.396078, 0.345098, 0.960784}
		\pgfsetstrokecolor{dialinecolor}
		\pgfsetstrokeopacity{1.000000}
		\definecolor{diafillcolor}{rgb}{0.396078, 0.345098, 0.960784}
		\pgfsetfillcolor{diafillcolor}
		\pgfsetfillopacity{1.000000}
		\node[anchor=base,inner sep=0pt, outer sep=0pt,color=dialinecolor] at (48.976600\du,15.440175\du){\tiny a-b};
		% setfont left to latex
		\definecolor{dialinecolor}{rgb}{0.396078, 0.345098, 0.960784}
		\pgfsetstrokecolor{dialinecolor}
		\pgfsetstrokeopacity{1.000000}
		\definecolor{diafillcolor}{rgb}{0.396078, 0.345098, 0.960784}
		\pgfsetfillcolor{diafillcolor}
		\pgfsetfillopacity{1.000000}
		\node[anchor=base,inner sep=0pt, outer sep=0pt,color=dialinecolor] at (51.037500\du,15.451675\du){\tiny b-a};
		% setfont left to latex
		\definecolor{dialinecolor}{rgb}{0.396078, 0.345098, 0.960784}
		\pgfsetstrokecolor{dialinecolor}
		\pgfsetstrokeopacity{1.000000}
		\definecolor{diafillcolor}{rgb}{0.396078, 0.345098, 0.960784}
		\pgfsetfillcolor{diafillcolor}
		\pgfsetfillopacity{1.000000}
		\node[anchor=base,inner sep=0pt, outer sep=0pt,color=dialinecolor] at (53.006800\du,15.451675\du){\tiny b+a};
		% setfont left to latex
		\definecolor{dialinecolor}{rgb}{0.000000, 0.000000, 0.000000}
		\pgfsetstrokecolor{dialinecolor}
		\pgfsetstrokeopacity{1.000000}
		\definecolor{diafillcolor}{rgb}{0.000000, 0.000000, 0.000000}
		\pgfsetfillcolor{diafillcolor}
		\pgfsetfillopacity{1.000000}
		\node[anchor=base,inner sep=0pt, outer sep=0pt,color=dialinecolor] at (54.525900\du,14.580517\du){\tiny t};
		% setfont left to latex
		\definecolor{dialinecolor}{rgb}{0.000000, 0.000000, 0.000000}
		\pgfsetstrokecolor{dialinecolor}
		\pgfsetstrokeopacity{1.000000}
		\definecolor{diafillcolor}{rgb}{0.000000, 0.000000, 0.000000}
		\pgfsetfillcolor{diafillcolor}
		\pgfsetfillopacity{1.000000}
		\node[anchor=base,inner sep=0pt, outer sep=0pt,color=dialinecolor] at (50.420800\du,10.820775\du){\tiny 2a};
		% setfont left to latex
		\definecolor{dialinecolor}{rgb}{0.396078, 0.345098, 0.960784}
		\pgfsetstrokecolor{dialinecolor}
		\pgfsetstrokeopacity{1.000000}
		\definecolor{diafillcolor}{rgb}{0.396078, 0.345098, 0.960784}
		\pgfsetfillcolor{diafillcolor}
		\pgfsetfillopacity{1.000000}
		\node[anchor=base,inner sep=0pt, outer sep=0pt,color=dialinecolor] at (50.558400\du,9.743345\du){\tiny y(t)};
		% setfont left to latex
		\definecolor{dialinecolor}{rgb}{0.000000, 0.000000, 0.000000}
		\pgfsetstrokecolor{dialinecolor}
		\pgfsetstrokeopacity{1.000000}
		\definecolor{diafillcolor}{rgb}{0.000000, 0.000000, 0.000000}
		\pgfsetfillcolor{diafillcolor}
		\pgfsetfillopacity{1.000000}
		\node[anchor=base,inner sep=0pt, outer sep=0pt,color=dialinecolor] at (41.836293\du,15.519991\du){\tiny a};
		\pgfsetlinewidth{0.060000\du}
		\pgfsetdash{}{0pt}
		\pgfsetbuttcap
		{
			\definecolor{diafillcolor}{rgb}{0.396078, 0.345098, 0.960784}
			\pgfsetfillcolor{diafillcolor}
			\pgfsetfillopacity{1.000000}
			% was here!!!
			\definecolor{dialinecolor}{rgb}{0.396078, 0.345098, 0.960784}
			\pgfsetstrokecolor{dialinecolor}
			\pgfsetstrokeopacity{1.000000}
			\draw (48.993499\du,11.015003\du)--(50.372353\du,11.015003\du);
		}
	\end{tikzpicture}
	\vspace{-2mm}
	\caption{\scriptsize Traslape segmento constante.}
	\label{fig:traslape2}
\end{figure}

\vspace{-9mm}
% Graphic for TeX using PGF
% Title: /home/jnicolaschc/GitHub/Teoría de Telecominicaciones I /ttl1_trabajo3/Documentos/desarrollo/codigofuente/pgf/trapezoide.dia
% Creator: Dia v0.97+git
% CreationDate: Sat Aug 21 00:36:32 2021
% For: jnicolaschc
% \usepackage{tikz}
% The following commands are not supported in PSTricks at present
% We define them conditionally, so when they are implemented,
% this pgf file will use them.
\begin{figure}[H]
	\centering
	\ifx\du\undefined
		\newlength{\du}
	\fi
	\setlength{\du}{15\unitlength}
	\begin{tikzpicture}[scale = 0.7]
		\pgftransformxscale{1.000000}
		\pgftransformyscale{-1.000000}
		\definecolor{dialinecolor}{rgb}{0.000000, 0.000000, 0.000000}
		\pgfsetstrokecolor{dialinecolor}
		\pgfsetstrokeopacity{1.000000}
		\definecolor{diafillcolor}{rgb}{1.000000, 1.000000, 1.000000}
		\pgfsetfillcolor{diafillcolor}
		\pgfsetfillopacity{1.000000}
		\pgfsetlinewidth{0.050000\du}
		\pgfsetdash{}{0pt}
		\pgfsetbuttcap
		{
			\definecolor{diafillcolor}{rgb}{0.000000, 0.000000, 0.000000}
			\pgfsetfillcolor{diafillcolor}
			\pgfsetfillopacity{1.000000}
			% was here!!!
			\pgfsetarrowsend{stealth}
			\definecolor{dialinecolor}{rgb}{0.000000, 0.000000, 0.000000}
			\pgfsetstrokecolor{dialinecolor}
			\pgfsetstrokeopacity{1.000000}
			\draw (44.008667\du,15.010903\du)--(56.001842\du,14.989880\du);
		}
		\pgfsetlinewidth{0.050000\du}
		\pgfsetdash{}{0pt}
		\pgfsetbuttcap
		{
			\definecolor{diafillcolor}{rgb}{0.000000, 0.000000, 0.000000}
			\pgfsetfillcolor{diafillcolor}
			\pgfsetfillopacity{1.000000}
			% was here!!!
			\pgfsetarrowsend{stealth}
			\definecolor{dialinecolor}{rgb}{0.000000, 0.000000, 0.000000}
			\pgfsetstrokecolor{dialinecolor}
			\pgfsetstrokeopacity{1.000000}
			\draw (50.005254\du,15.000392\du)--(49.993900\du,9.052460\du);
		}
		\pgfsetlinewidth{0.050000\du}
		\pgfsetdash{{0.300000\du}{0.120000\du}{0.060000\du}{0.120000\du}}{0cm}
		\pgfsetbuttcap
		{
			\definecolor{diafillcolor}{rgb}{0.101961, 0.682353, 0.623529}
			\pgfsetfillcolor{diafillcolor}
			\pgfsetfillopacity{1.000000}
			% was here!!!
			\definecolor{dialinecolor}{rgb}{0.101961, 0.682353, 0.623529}
			\pgfsetstrokecolor{dialinecolor}
			\pgfsetstrokeopacity{1.000000}
			\draw (46.996400\du,15.021100\du)--(47.008100\du,9.063880\du);
		}
		\pgfsetlinewidth{0.050000\du}
		\pgfsetdash{{0.300000\du}{0.120000\du}{0.060000\du}{0.120000\du}}{0cm}
		\pgfsetbuttcap
		{
			\definecolor{diafillcolor}{rgb}{0.101961, 0.682353, 0.623529}
			\pgfsetfillcolor{diafillcolor}
			\pgfsetfillopacity{1.000000}
			% was here!!!
			\definecolor{dialinecolor}{rgb}{0.101961, 0.682353, 0.623529}
			\pgfsetstrokecolor{dialinecolor}
			\pgfsetstrokeopacity{1.000000}
			\draw (48.980200\du,15.007400\du)--(48.992000\du,9.050250\du);
		}
		\pgfsetlinewidth{0.050000\du}
		\pgfsetdash{{0.300000\du}{0.120000\du}{0.060000\du}{0.120000\du}}{0cm}
		\pgfsetbuttcap
		{
			\definecolor{diafillcolor}{rgb}{0.101961, 0.682353, 0.623529}
			\pgfsetfillcolor{diafillcolor}
			\pgfsetfillopacity{1.000000}
			% was here!!!
			\definecolor{dialinecolor}{rgb}{0.101961, 0.682353, 0.623529}
			\pgfsetstrokecolor{dialinecolor}
			\pgfsetstrokeopacity{1.000000}
			\draw (51.022200\du,14.980200\du)--(51.033900\du,9.022980\du);
		}
		\pgfsetlinewidth{0.050000\du}
		\pgfsetdash{{0.300000\du}{0.120000\du}{0.060000\du}{0.120000\du}}{0cm}
		\pgfsetbuttcap
		{
			\definecolor{diafillcolor}{rgb}{0.101961, 0.682353, 0.623529}
			\pgfsetfillcolor{diafillcolor}
			\pgfsetfillopacity{1.000000}
			% was here!!!
			\definecolor{dialinecolor}{rgb}{0.101961, 0.682353, 0.623529}
			\pgfsetstrokecolor{dialinecolor}
			\pgfsetstrokeopacity{1.000000}
			\draw (52.996000\du,14.980200\du)--(53.007700\du,9.022980\du);
		}
		\pgfsetlinewidth{0.060000\du}
		\pgfsetdash{}{0pt}
		\pgfsetbuttcap
		{
			\definecolor{diafillcolor}{rgb}{0.396078, 0.345098, 0.960784}
			\pgfsetfillcolor{diafillcolor}
			\pgfsetfillopacity{1.000000}
			% was here!!!
			\definecolor{dialinecolor}{rgb}{0.396078, 0.345098, 0.960784}
			\pgfsetstrokecolor{dialinecolor}
			\pgfsetstrokeopacity{1.000000}
			\draw (44.028502\du,15.015292\du)--(46.988200\du,15.011200\du);
		}
		\pgfsetlinewidth{0.060000\du}
		\pgfsetdash{}{0pt}
		\pgfsetbuttcap
		{
			\definecolor{diafillcolor}{rgb}{0.396078, 0.345098, 0.960784}
			\pgfsetfillcolor{diafillcolor}
			\pgfsetfillopacity{1.000000}
			% was here!!!
			\definecolor{dialinecolor}{rgb}{0.396078, 0.345098, 0.960784}
			\pgfsetstrokecolor{dialinecolor}
			\pgfsetstrokeopacity{1.000000}
			\draw (47.011200\du,14.999700\du)--(48.992089\du,10.999108\du);
		}
		% setfont left to latex
		\definecolor{dialinecolor}{rgb}{0.396078, 0.345098, 0.960784}
		\pgfsetstrokecolor{dialinecolor}
		\pgfsetstrokeopacity{1.000000}
		\definecolor{diafillcolor}{rgb}{0.396078, 0.345098, 0.960784}
		\pgfsetfillcolor{diafillcolor}
		\pgfsetfillopacity{1.000000}
		\node[anchor=base,inner sep=0pt, outer sep=0pt,color=dialinecolor] at (47.009500\du,15.451675\du){\tiny -(a+b)};
		% setfont left to latex
		\definecolor{dialinecolor}{rgb}{0.396078, 0.345098, 0.960784}
		\pgfsetstrokecolor{dialinecolor}
		\pgfsetstrokeopacity{1.000000}
		\definecolor{diafillcolor}{rgb}{0.396078, 0.345098, 0.960784}
		\pgfsetfillcolor{diafillcolor}
		\pgfsetfillopacity{1.000000}
		\node[anchor=base,inner sep=0pt, outer sep=0pt,color=dialinecolor] at (48.976600\du,15.440175\du){\tiny a-b};
		% setfont left to latex
		\definecolor{dialinecolor}{rgb}{0.396078, 0.345098, 0.960784}
		\pgfsetstrokecolor{dialinecolor}
		\pgfsetstrokeopacity{1.000000}
		\definecolor{diafillcolor}{rgb}{0.396078, 0.345098, 0.960784}
		\pgfsetfillcolor{diafillcolor}
		\pgfsetfillopacity{1.000000}
		\node[anchor=base,inner sep=0pt, outer sep=0pt,color=dialinecolor] at (51.037500\du,15.451675\du){\tiny b-a};
		% setfont left to latex
		\definecolor{dialinecolor}{rgb}{0.396078, 0.345098, 0.960784}
		\pgfsetstrokecolor{dialinecolor}
		\pgfsetstrokeopacity{1.000000}
		\definecolor{diafillcolor}{rgb}{0.396078, 0.345098, 0.960784}
		\pgfsetfillcolor{diafillcolor}
		\pgfsetfillopacity{1.000000}
		\node[anchor=base,inner sep=0pt, outer sep=0pt,color=dialinecolor] at (53.006800\du,15.451675\du){\tiny b+a};
		% setfont left to latex
		\definecolor{dialinecolor}{rgb}{0.000000, 0.000000, 0.000000}
		\pgfsetstrokecolor{dialinecolor}
		\pgfsetstrokeopacity{1.000000}
		\definecolor{diafillcolor}{rgb}{0.000000, 0.000000, 0.000000}
		\pgfsetfillcolor{diafillcolor}
		\pgfsetfillopacity{1.000000}
		\node[anchor=base,inner sep=0pt, outer sep=0pt,color=dialinecolor] at (55.525900\du,14.580517\du){\tiny t};
		% setfont left to latex
		\definecolor{dialinecolor}{rgb}{0.000000, 0.000000, 0.000000}
		\pgfsetstrokecolor{dialinecolor}
		\pgfsetstrokeopacity{1.000000}
		\definecolor{diafillcolor}{rgb}{0.000000, 0.000000, 0.000000}
		\pgfsetfillcolor{diafillcolor}
		\pgfsetfillopacity{1.000000}
		\node[anchor=base,inner sep=0pt, outer sep=0pt,color=dialinecolor] at (50.420800\du,10.820775\du){\tiny 2a};
		% setfont left to latex
		\definecolor{dialinecolor}{rgb}{0.396078, 0.345098, 0.960784}
		\pgfsetstrokecolor{dialinecolor}
		\pgfsetstrokeopacity{1.000000}
		\definecolor{diafillcolor}{rgb}{0.396078, 0.345098, 0.960784}
		\pgfsetfillcolor{diafillcolor}
		\pgfsetfillopacity{1.000000}
		\node[anchor=base,inner sep=0pt, outer sep=0pt,color=dialinecolor] at (50.558400\du,9.743345\du){\tiny y(t)};
		\pgfsetlinewidth{0.060000\du}
		\pgfsetdash{}{0pt}
		\pgfsetbuttcap
		{
			\definecolor{diafillcolor}{rgb}{0.396078, 0.345098, 0.960784}
			\pgfsetfillcolor{diafillcolor}
			\pgfsetfillopacity{1.000000}
			% was here!!!
			\definecolor{dialinecolor}{rgb}{0.396078, 0.345098, 0.960784}
			\pgfsetstrokecolor{dialinecolor}
			\pgfsetstrokeopacity{1.000000}
			\draw (49.004589\du,11.005358\du)--(51.017077\du,11.011608\du);
		}
		\pgfsetlinewidth{0.060000\du}
		\pgfsetdash{}{0pt}
		\pgfsetbuttcap
		{
			\definecolor{diafillcolor}{rgb}{0.396078, 0.345098, 0.960784}
			\pgfsetfillcolor{diafillcolor}
			\pgfsetfillopacity{1.000000}
			% was here!!!
			\definecolor{dialinecolor}{rgb}{0.396078, 0.345098, 0.960784}
			\pgfsetstrokecolor{dialinecolor}
			\pgfsetstrokeopacity{1.000000}
			\draw (52.992709\du,14.994164\du)--(51.017077\du,11.017858\du);
		}
		\pgfsetlinewidth{0.060000\du}
		\pgfsetdash{}{0pt}
		\pgfsetbuttcap
		{
			\definecolor{diafillcolor}{rgb}{0.396078, 0.345098, 0.960784}
			\pgfsetfillcolor{diafillcolor}
			\pgfsetfillopacity{1.000000}
			% was here!!!
			\definecolor{dialinecolor}{rgb}{0.396078, 0.345098, 0.960784}
			\pgfsetstrokecolor{dialinecolor}
			\pgfsetstrokeopacity{1.000000}
			\draw (52.996740\du,15.000650\du)--(55.356049\du,14.988030\du);
		}
	\end{tikzpicture}
	\vspace{-2mm}
	\caption{\scriptsize Señal resultante de la convolución.}
	\label{fig:trapezoide}
\end{figure}

\vspace{-5mm}

El siguiente paso fue generar este proceso en el desarrollo computacional con el uso de dos funciones tipo seno cardinal en el dominio del tiempo y realizar el proceso de convolución para obtener la respuesta en la frecuencia trapezoidal. Para el proceso de filtrar la señal se aplicó la Transformada de Fourier a la señal obtenida de la convolución de estas dos funciones sinc y filtrar la respuesta en frecuencia de la señal asignada, para observar los efectos del filtro en la señal se realizó la Transformada Inversa de Fourier al resultado obtenido. Después de obtener los resultados necesarios se continuó calculando la densidad espectral de energía en la señal original y también a las señales resultantes después de haberles aplicado los filtros en cada uno de los puntos, para realizar análisis de estos resultados se tuvo en cuenta la comparación porcentual entre la energía de la señal original y las resultantes \cite{densidad2021}, además se analizo la presencia de distorsión en los resultados a partir de las condiciones que plantea la teoría de distorsión lineal \cite{quiz2021}.

\subsection{Plan de pruebas}\label{planPruebas}
Después de haber realizado el planteamiento de cómo desarrollar y obtener los resultados en el script, se procedió a realizar la planificación de las pruebas de funcionamiento del código con el objetivo de verificar y analizar los resultados a través de algunos escenarios que se definen a continuación:

\begin{itemize}
	\item Aumentar y disminuir el rango de frecuencias del filtro ideal.
	\item Variar el rango de frecuencias del filtro de trapezoidal donde su respuesta es constante.
\end{itemize}

Con los escenarios proyectados anteriormente se espera obtener resultados que permitan evidenciar los fenómenos estudiados. Además se plantean unos objetivos claves para verificar y analizar en los resultados la presencia de los teoremas relacionados con la temática de Sistemas LTI, los cuales se describen a continuación:

\begin{enumerate}
	\item Análisis de los efectos de aumentar el ancho de la banda de paso del filtro ideal.
	\item Análisis de los efectos de disminuir el ancho de la banda de paso del filtro ideal.
	\item Análisis de los efectos de variar el rango de frecuencias del filtro trapezoidal donde su respuesta es constante.
	\item Análisis de resultados a partir del teorema de la energía de Rayleigh.
	\item Análisis de los resultados a partir de la teoría de distorsión lineal.
\end{enumerate}
